\documentclass[10pt,letterpaper]{article}
\usepackage[utf8]{inputenc}
\usepackage{amsmath}
\usepackage{amsfonts}
\usepackage{amssymb}
\usepackage{graphicx}
\usepackage{bbm}
\usepackage{listings}
\usepackage{url}
\usepackage{amsopn,amssymb,thmtools,thm-restate}
\usepackage{algorithm}
\usepackage{algorithmic}
\usepackage{amsopn,amssymb,thmtools,thm-restate}
\usepackage{amsthm}
\usepackage{algorithm}
\usepackage{algorithmic}

\DeclareMathOperator{\argmax}{arg\,max}
\DeclareMathOperator{\argmin}{arg\,min}
\DeclareMathOperator{\argsort}{arg\,sort}
\newcommand{\tree}{\mathcal{T}}
\newcommand{\xopt}{x^*}
\newcommand{\robot}{R}
\newcommand{\obst}{F}
\newcommand{\obj}{B}
\newcommand{\cspace}{C}
\newcommand{\state}{s}
\newcommand{\action}{a}
\newcommand{\statespace}{S}
\newcommand{\actionspace}{A}
\newcommand{\traj}{\tau}
\newcommand{\op}{\mathfrak{o}}
\newcommand{\Op}{\mathcal{O}}
\newcommand{\inp}{\alpha}
\newcommand{\policy}{\pi}
\newcommand{\plpolicy}{\pi_{\text{pl}}}
\newcommand{\disc}{\delta}
\newcommand{\discspace}{\Delta}
\newcommand{\cont}{\kappa}
\newcommand{\contspace}{K}
\newcommand{\solop}{\contspace^*_s }
\newcommand{\Dpl}{\mathbf{D}_\text{pl}}
\newcommand{\Drl}{\mathbf{D}_\text{RL}}
\newcommand{\opol}{\pi_{i}}
\newcommand{\aw}{a^{\op}}
\newcommand{\hQ}{\hat{Q}_{\alpha}}
\newcommand{\regQ}{\mathcal{R}(\hQ)}
\newcommand{\Ppl}{P_{pl}}
\newcommand{\Ppi}{P_{\pi}}
\newcommand{\good}{\epsilon}
\newcommand{\prob}{\omega}
\newcommand{\probfeature}{\phi}
\newcommand{\admon}{{\sc AdMon}}
\newcommand{\gtamp}{G-TAMP}
\newcommand{\ddpg}{{\sc ddpg}}
\newcommand{\ppo}{{\sc ppo}}
\newcommand{\gail}{{\sc gail}}
\newcommand{\region}{\mathcal{R}}
\newcommand{\motion}{\tau}
\newcommand{\pickconf}{c_{pick}}
\newcommand{\placeconf}{c_{place}}
\newcommand{\initconf}{c_{init}}

\newcommand{\trj}{operator sequence}
\newcommand{\tr}{sequence}
\newcommand{\trjs}{operator sequences}
\newcommand{\trs}{sequences}

\newcommand{\workspace}{\mathcal{W}}


\author{Beomjoon Kim}
\title{Continuous search algorithm}
\begin{document}
\maketitle
\section{Introduction}
Motivation
\begin{itemize}
\item Planning in a continuous state-action space is important in robotics
\item Extending heuristic search algorithms, such as A*, with sampling
to deal with continuous search space is viable but requires
a good method to sample from the action space, and a good heuristic
function. Also, in the discrete case such class of algorithms may
be able to find a shortest-path, assuming the heuristic function
is admissible, but it is not true true in the continuous case.
\item Monte Carlo planning approaches offer a way to estimate the
heuristic function
\item We use a variant 
\item 
\end{itemize}


\section{Problem formulation}
Given a computational budget B,
a plan skeleton $\{\op_1(\disc_1,\cdot),\cdots,\op_T(\disc_t,\cdot)\}$, 
and the set of goal states $S_G$, possibly described with a predicate,
find the continuous parameters $\cont_1,\cdots,\cont_T$ such that it maximizes
the sum of the discounted rewards,
$$ \max_{\cont_1,\cdots,\cont_T}  \sum_{t=0}^{T
}  \gamma^t r(s_t,\cont_t)  $$
where $r(s_t,\cont_t) = r(s_t,\op_t(\disc_t,\cont_t) )$, $s_T \in S_G$ and
the generative model of the environment $T$ such that $s_{t+1} = T(s_t,\op_t(\disc_t,\cont_T))$





\bibliographystyle{IEEEtran}
\bibliography{references}
\end{document}
