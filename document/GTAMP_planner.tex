\documentclass[10pt,letterpaper]{article}
\usepackage[utf8]{inputenc}
\usepackage{amsmath}
\usepackage{amsfonts}
\usepackage{amssymb}
\usepackage{graphicx}
\usepackage{bbm}
\usepackage{listings}
\usepackage{url}
\usepackage{amsopn,amssymb,thmtools,thm-restate}
\usepackage{algorithm}
\usepackage{algorithmic}
\usepackage{amsopn,amssymb,thmtools,thm-restate}
\usepackage{amsthm}
\usepackage{algorithm}
\usepackage{algorithmic}

\DeclareMathOperator{\argmax}{arg\,max}
\DeclareMathOperator{\argmin}{arg\,min}
\DeclareMathOperator{\argsort}{arg\,sort}
\newcommand{\tree}{\mathcal{T}}
\newcommand{\xopt}{x^*}
\newcommand{\robot}{R}
\newcommand{\obst}{F}
\newcommand{\obj}{B}
\newcommand{\cspace}{C}
\newcommand{\state}{s}
\newcommand{\action}{a}
\newcommand{\statespace}{S}
\newcommand{\actionspace}{A}
\newcommand{\traj}{\tau}
\newcommand{\op}{\mathfrak{o}}
\newcommand{\Op}{\mathcal{O}}
\newcommand{\inp}{\alpha}
\newcommand{\policy}{\pi}
\newcommand{\plpolicy}{\pi_{\text{pl}}}
\newcommand{\disc}{\delta}
\newcommand{\discspace}{\Delta}
\newcommand{\cont}{\kappa}
\newcommand{\contspace}{K}
\newcommand{\solop}{\contspace^*_s }
\newcommand{\Dpl}{\mathbf{D}_\text{pl}}
\newcommand{\Drl}{\mathbf{D}_\text{RL}}
\newcommand{\opol}{\pi_{i}}
\newcommand{\aw}{a^{\op}}
\newcommand{\hQ}{\hat{Q}_{\alpha}}
\newcommand{\regQ}{\mathcal{R}(\hQ)}
\newcommand{\Ppl}{P_{pl}}
\newcommand{\Ppi}{P_{\pi}}
\newcommand{\good}{\epsilon}
\newcommand{\prob}{\omega}
\newcommand{\probfeature}{\phi}
\newcommand{\admon}{{\sc AdMon}}
\newcommand{\gtamp}{G-TAMP}
\newcommand{\ddpg}{{\sc ddpg}}
\newcommand{\ppo}{{\sc ppo}}
\newcommand{\gail}{{\sc gail}}
\newcommand{\region}{\mathcal{R}}
\newcommand{\motion}{\tau}
\newcommand{\pickconf}{c_{pick}}
\newcommand{\placeconf}{c_{place}}
\newcommand{\initconf}{c_{init}}

\newcommand{\trj}{operator sequence}
\newcommand{\tr}{sequence}
\newcommand{\trjs}{operator sequences}
\newcommand{\trs}{sequences}

\newcommand{\workspace}{\mathcal{W}}


\author{Beomjoon Kim}
\title{A planning algorithm for G-TAMP}
\begin{document}
\maketitle
\section{Introduction}
\emph{Motivation:}
\begin{itemize}
\item TAMP planners integrate discrete symbolic reasoning with continuous
geometric reasoning to plan robot motions to achieve a high-level objective.
\item While TAMP planners need make decisions at different levels of abstractions,
for a large class of TAMP problems, we need to
eventually solve a sequence of reaching-and-packing problems. For instance, consider
the task of cooking a meal. We may, at
a purely symbolic level, make a plan that says the target raw food
should be washed first, and then placed on a stove on top of a frying pan, without
performing any geometric reasoning. At a more concrete level, however,
the problem eventually comes down to finding the motions for reaching and
placing it in the region determined
in the task plan. All the while, if there are any obstacles that are
blocking the path of the reaching-and-packing motion, the planner also need
to identify these objects and find motions to clear them. 
\item What makes such geometric reasoning challenging is the intricate
relationship among individual reaching-and-packing problems: how we clear
obstacles for reaching-and-placing the raw food can influence how we
reach-and-place the frying pan.
\item We define such problems as geometric TAMP problems. Our goal is to
first formally formulate this problem, and identify the structures of these problems
to propose a planning algorithm that exploits such structures to solve them
efficiently and optimally. Our algorithm can be used as an interface layer  used
to check the feasibility of the task-plan, much like how existing
TAMP planners use motion planners such as RRT to check the 
feasibility of individual pick and place operator
instances.
\item We focus on mobile-manipulation problems where, feasibility check
might come down to calling a path planner with an infeasible problem instance,
rather than simply checking existence of an inverse kinematics solution
and a collision at the end-configuration which in general takes
faster than reporting failure with a sampling-based path planner.
\end{itemize}
Informal problem formulation:
\begin{itemize}
\item Specifically, given a task-plan that consists of 
the sequence of abstract and discrete operators which
satisfies the task-level goal, our objective is to find the continuous 
decision variables such as grasps, base poses, placements, 
and collision-free paths that correspond to the given
abstract operator sequence, as well as any obstacles that are in the way, and
the associated continuous variables to clear them, such that the solution
optimizes the objective function.
\item This is a long-horizon problem with an infinite branching factor and a set
of hybrid decision variables. We
could, in principle, use uniform-random sampling to sample operator instances
and heuristic-forward search that reconsiders past nodes, 
to find a solution. However,
this algorithm is inefficient in that it cannot determine which branch to re-explore,
or which node to back-track to. Moreover, as we will show,
uniform sampling cannot find an optimal solution.
\end{itemize}
Algorithmic contribution:
\begin{itemize}
\item We offer an anytime planning algorithm, which, in similar spirit to
Monte-Carlo Tree Search, uses the online estimate of the heuristic
to decide which branch to re-explore in a principled manner. Our algorithm
operates on different levels of hierarchy which effectively 
shortens the planning horizon, and allows efficient back-tracking. To determine
the obstacles to clear, we use the swept-volume of reaching and packing motions
for the key objects, rather than random sampling.
To optimize the heuristic function, we propose a novel sampling strategy that
not only finds a feasible solution but also an optimal solution under
mild assumptions.
\item Specifically, our algorithm operates at three different levels of hierarchy.
In the highest level, we have the task-planner that outputs the abstract
task plan. This task plan is then passed on to the next layer.
In the second layer, we need solve the \emph{reaching} and \emph{packing}
problems in which we decide how we will fetch the given object, by finding grasps
and pick base pose, as well as how to pack the object into a region,
by finding object placement, placement base pose,
and motions from the pick configuration to the place configuration,
ignoring any obstacles in the way. The swept-volume of the fetching and packing
motion is then passed into the layer below, which then solves an instance
of the \emph{navigation-among-movable-obstacles (NAMO)} problem. At
this stage, the abstract plan becomes completely concrete, and the
planner outputs the sequence of configurations that clears obstacles
and fetches and packs the target object. We then move onto the next abstract 
operator in the task plan.

\iffalse
\item To gain planning efficiency to find the first feasible solution
yet optimize the plan, we take the finding-and-committing 
approach: we first find a feasible solution for each step in the task-plan
 and each level of hierarchy and commit to it without back-tracking across 
them, and move onto the next. This will not affect correctness of our
algorithm, because while how we fetch, pack, and clear obstacles
for the first object in the task-plan may influence the
optimality for the next plan, but it will not affect feasibility.

At each level of hierarchy and each step in the task plan, the latter
part of the planning problem cannot be defined until we solve the
earlier part. For instance, we cannot solve the NAMO problem associated
with fetching and packing an object, because we do not have a swept-volume
yet. We also cannot move onto the 
\item What do I want to say next? Introduce MCTS. Introduce the idea
of committing and then optimizing. We have reward functions.
We have 
\item  
\fi
\end{itemize}

\iffalse
Some benefits in terms of algorithm usage:
\begin{itemize}
\item Unlike previous TAMP planners that use symbolic planners, such as 
FastDownward, to perform geometric reasoning using predicates such as
Reachable, we assume that the symbolic planner only has access to abstract
reaching motion, such as pick, or placing motion, such as place or push,
whose preconditions and effects only consists of discrete symbolic 
predicates. This allows us to write a simpler task-plan domain description.
\end{itemize}

\section{Problem formulation}
We assume that the task-planner is given abstract reaching and placing
operators. They are abstract in the sense that their inputs are only discrete 
elements in the world, such as objects and regions. While in general
there are other useful non-geometric operators, such as wiping a dish,
we will focus our attention to geometric operators that require 
feasibility checks using a call to a motion planner.

The task-plan would then consists of a sequence of abstract pick-and-places.
We assume that the task-planner is cost-efficient in the sense that it will
not touch objects that are not relevant in solving the problem. 

More formally, we assume that we have a set of regions, $\{\region_1,\cdots,\region_m\}$,
a set of objects, $\{\obj_1,\cdots,\obj_n\}$, and a set of concrete pick and place
operators and their samplers. 
The task-plan is given in the following form, as a sequence
of abstract pick-and-places.
$$[PaP(\obj_1, \region_1), \cdots PaP(\obj_T, \region_T))]$$
Our goal is to find a sequence of configurations
that achieve these picks and places.

\section{Our approach}
Hierarchical planning and limited horizon planning for NAMO problems, with Monte-Carlo
planning.

\subsection{Receding-horizon NAMO planning}
Based on the description of our problem, we can formulate the NAMO problem
as follows. Given $\initconf, \pickconf, \placeconf$ and $\motion_{place}$, 
where $\pickconf$ and 
$\placeconf$  are beginning and end of $\motion_{place}$ find a sequence of 
robot motions from $\initconf$ to clear obstacles present in $\motion_{place}$. 
We have the following algorithm, that generalizes Stillman's algorithm to continuous
search space.
\begin{enumerate}
\item Plan a path from current configuration to $\pickconf$, ignoring obstacles.
Define this as $\motion_{pick}$
\item If any, plan a pick-and-place motion to clear the first obstacle in 
$\motion_{pick}$. Make sure to not to place it in $\motion_{place}$. 
\item Repeat 1 and 2 until there are no obstacles in $\motion_{pick}$.
\item If any, plan a pick-and-place motion to clear the first obstacle in 
$\motion$. Make sure to not to place it in $\motion_{pick}$. Repeat until
no obstacles are in the way.
\end{enumerate}
Now, even for getting a single object, this may be a long-horizon planning problem,
and it is non-trivial to which stage we should back-track to, if the current pick or
placement turns out to be infeasible. So, we propose to use a
 receding-horizon approach, by limiting
the NAMO problem into different regions, when available,
 and limit the clearance of obstacles within
that region.  The underlying assumption is that the NAMO problem within a region
can be solved by rearranging obstacles within that region. Sometimes this is
true by the description of the problem - two regions are defined as separate
regions because they are used to store obstacles that are different sizes,
or because these regions are too far (ex. loading truck and packing area).
Sometimes, it is not necessarily true at all (ex. my kitchen).

\subsection{Monte Carlo Planning}
Monte-Carlo tree search with progressive widening.
\begin{algorithm}[htb]
\small
   \caption{\sc{search}($s_0$)}
   \label{alg:search}
\begin{algorithmic}[1]
\STATE $\tree(s_0) = \{ N(s_0)=1,A=\emptyset,Q(s_0,\cdot)=0,N(s_0,\cdot)=0 \}$
\REPEAT
\STATE {\sc simulate}($s_0,0$)
\UNTIL{$timeout$}
\STATE{\bf return} $\argmax_{a\in\tree(s_0).A} \tree(s_0).Q(s,a)$
\end{algorithmic}
\end{algorithm}

\begin{algorithm}[htb]
\small
   \caption{\sc{simulate}($s, T, \delta, \alpha, \epsilon, S_G$)}
   \label{alg:simulate}
\begin{algorithmic}[1]
\IF{$s \in S_G$}
\STATE{\bf return} \text{success-rwd}
\ENDIF
\IF{$|\tree(s).A| \leq \delta \cdot \tree(s).N(s)$}
\STATE //progressive widening step
\STATE $a \sim ${\sc SAMPLE}($\tree(s).A, \epsilon$)// \text{based on a sampling algorithm}
\STATE $\tree(s).A = \tree(s).A \cup a$
\ELSE
\STATE $a = \argmax_{a \in tree(s).A} Q(s,a, \alpha)$
\ENDIF
\IF{$a\ is\ infeasible$}
\STATE{\bf return} \text{infeasible-rwd}
\ENDIF
\STATE $s',r ~ \sim f(s,a)$
\STATE $R = r + \gamma \cdot $\sc{simulate}$(s',depth+1)$
\STATE $\tree(s).N(s,a) += 1$
\STATE $\tree(s).Q(s,a) = Q(s,a) + \frac{R-\tree(s).Q(s,a)}{\tree(s).N(s,a)}$
\end{algorithmic}
\end{algorithm}

\section{Limitations}
\subsection{Limitation of using abstract operators during task-planning stage}
The problem with this is that, we cannot know, during the 
task-planning stage, that a NAMO plan might move an 
essential object out of its goal pose for NAMO purpose. 
For example, in trying to pick the target object located 
in the sink, we might have moved the detergent out of its 
"detergent pose", to which it must be placed back to. However, 
if the detergent was at the detergent pose in the initial state, 
the task planner would not recognize that it can go out of its this pose.

One possible way to resolve this might be to add a side-effect to the 
abstract pick and place operators, that says, the poses of objects 
that are in the same region as the object being picked might change. 
In HPN, what was the purpose of side effects, and how did you end 
up using them? Was it for making the search complete or making the planner sound?

\subsection{Limitation of using swept volume}
Even in LP1, we cannot move obstacles out of the way that are not detected in
the swept volume.

\iffalse
\section{Geometric task-and-motion-planning problem}
We denote the workspace of the robot as $\workspace$. A region $\region$ is
defined as a subset of the workspace. We define \emph{task constraint} as a set of objects 
 to be packed in the ordered sequence of regions,
$$[\region_1:(\obj^{(1)}_1,\cdots,\obj^{(1)}_{m_1}), \cdots, 
\region_T:(\obj^{(T)}_1,\cdots,\obj^{(T)}_{m_T})]$$ and the \emph{connecting region}
$\region_{connect}$ between two regions $\region_i$ and $\region_j$ as
$\region_{connect} \not\subset \region_i \cup \region_j$ such that,
 from any robot configuration in $\region_{connect}$, any configuration
in  $\region_i$ and $\region_j$ can be reached, 
without considering the movable obstacles. We will call the regions
specified in the task constraint as \emph{key regions}.


The problem statement is as follows. \emph{Given a task constraint, regions, and connecting 
regions between all pairs of regions, find the optimal motion plan to 
pack each object into its corresponding
key region in the given order, where the quality of a plan is measured by the number of
robot operations in it.}  We have following variability in our problem:
\begin{itemize}
\item The objects specified within a key region may or may not be ordered.
\item There may be other movable obstacles that are not specified in the task constraint.
\end{itemize}
We have following assumptions:
\begin{itemize}
\item We cannot find a feasible solution of the packing problem for $\region_{t}$ without
finding one for $\region_{t-1}$.
\item How we pack objects in $\region_{t-1}$ does not affect the
feasibility of the packing problem for $\region_{t}$, but affects the optimality.
\end{itemize}

Given this setup, G-TAMP problem is a sequence of interrelated packing problems. 
A packing problem
$$\mathcal{R}: (\obj_1,\cdots,\obj_{m})$$
consists of two parts: fetching objects from their initial regions, and
then placing them in $\region$. We assume that we cannot move a target
object once we pack it in $\region$, in order to remove redundant actions.

In fetching-and-placing a target object, we need to determine
objects that are in the way and clear them out. 
Therefore, we can see that a packing problem has
four unknowns, for each object:
\begin{enumerate}
\item Finding a robot reaching motion to the current target object
from the robot's current configuration. 
\item Finding a robot constraint-removal motion for clearing movable obstacles
from this reaching-motion
\item Finding a robot placing motion that places the current target
object in $\region$ such that all of target objects $\obj_1,\cdots,\obj_m$ can
be packed in $\region$.
\item Finding a robot constraint-removal motion for clearing movable obstacles
from this placing-motion
\end{enumerate}

These problems are related in that we cannot solve for 2 without solving for
1, 3 without solving for 2, and so on. However, if we tackle this naively in this 
order, we may waste our effort finding solutions 
for 1, 2, and 3 for many early objects, only to find 
that these early placements prevents a solution for latter objects. 


How should we define the sub-problem for which MCTS will be used for?
We should define it such that a solution to the earlier sub-problem
cannot have feasibility-level influence on the solution of the
next sub-problem, and we need the solution to the sub-problem
for the subsequent sub-problem. For example, we can separate object
fetching problem and NAMO problem, because how I fetch objects
cannot influence the feasibility of the NAMO problem, and
I cannot solve (or even define) the NAMO problem without knowing the fetching path.
But I cannot separate clearing each object within a NAMO problem
because if I clear an earlier obstacle such that it makes the
next obstacle unreachable, then the problem is infeasible.


I cannot define the reaching-problem for object $B_2$ if I have not
solved the reaching-and-constraint-removal motion for object $B_1$,
because I do not know how objects will be arranged. Also, I am assuming (?)
that no matter how I clear obstacles and reach-and-pack object $B_1$,
it cannot influence the feasiblity of reach-and-place of object $B_2$,
because at the end of clearing-reaching-and-packing, I will be at the
connecting region to $B_2$, because I ensure an existence of path
to a point in connecting region of $B_2$.

What's the connecting region between the box region and the home region?
What if I have to get stuff from the kitchen and shelf to box 1?
\fi
\fi

\bibliographystyle{IEEEtran}
\bibliography{references}
\end{document}
