\documentclass[10pt,letterpaper]{article}
\usepackage[utf8]{inputenc}
\usepackage{amsmath}
\usepackage{amsfonts}
\usepackage{amssymb}
\usepackage{graphicx}
\usepackage{bbm}
\usepackage{listings}
\usepackage{url}
\usepackage{amsopn,amssymb,thmtools,thm-restate}
\usepackage{algorithm}
\usepackage{algorithmic}
\usepackage{amsopn,amssymb,thmtools,thm-restate}
\usepackage{amsthm}
\usepackage{algorithm}
\usepackage{algorithmic}

\DeclareMathOperator{\argmax}{arg\,max}
\DeclareMathOperator{\argmin}{arg\,min}
\DeclareMathOperator{\argsort}{arg\,sort}
\newcommand{\tree}{\mathcal{T}}
\newcommand{\xopt}{x^*}
\newcommand{\robot}{R}
\newcommand{\obst}{F}
\newcommand{\obj}{B}
\newcommand{\cspace}{C}
\newcommand{\state}{s}
\newcommand{\action}{a}
\newcommand{\statespace}{S}
\newcommand{\actionspace}{A}
\newcommand{\traj}{\tau}
\newcommand{\op}{\mathfrak{o}}
\newcommand{\Op}{\mathcal{O}}
\newcommand{\inp}{\alpha}
\newcommand{\policy}{\pi}
\newcommand{\plpolicy}{\pi_{\text{pl}}}
\newcommand{\disc}{\delta}
\newcommand{\discspace}{\Delta}
\newcommand{\cont}{\kappa}
\newcommand{\contspace}{K}
\newcommand{\solop}{\contspace^*_s }
\newcommand{\Dpl}{\mathbf{D}_\text{pl}}
\newcommand{\Drl}{\mathbf{D}_\text{RL}}
\newcommand{\opol}{\pi_{i}}
\newcommand{\aw}{a^{\op}}
\newcommand{\hQ}{\hat{Q}_{\alpha}}
\newcommand{\regQ}{\mathcal{R}(\hQ)}
\newcommand{\Ppl}{P_{pl}}
\newcommand{\Ppi}{P_{\pi}}
\newcommand{\good}{\epsilon}
\newcommand{\prob}{\omega}
\newcommand{\probfeature}{\phi}
\newcommand{\admon}{{\sc AdMon}}
\newcommand{\gtamp}{G-TAMP}
\newcommand{\ddpg}{{\sc ddpg}}
\newcommand{\ppo}{{\sc ppo}}
\newcommand{\gail}{{\sc gail}}
\newcommand{\region}{\mathcal{R}}
\newcommand{\motion}{\tau}
\newcommand{\pickconf}{c_{pick}}
\newcommand{\placeconf}{c_{place}}
\newcommand{\initconf}{c_{init}}

\newcommand{\trj}{operator sequence}
\newcommand{\tr}{sequence}
\newcommand{\trjs}{operator sequences}
\newcommand{\trs}{sequences}

\newcommand{\workspace}{\mathcal{W}}


\author{Beomjoon Kim}
\title{Feedback-based Continuous planning for G-TAMP}
\begin{document}
\maketitle
\section{Problem formulation}
Given a computational budget B,
a plan skeleton $\{\op_1(\disc_1,\cdot),\cdots,\op_T(\disc_t,\cdot)\}$, 
and the set of goal states $S_G$, possibly described with a predicate,
find the continuous parameters $\cont_1,\cdots,\cont_T$ such that it maximizes
the sum of the discounted rewards,
$$ \max_{\cont_1,\cdots,\cont_T}  \sum_{t=0}^{T
}  \gamma^t r(s_t,\cont_t)  $$
where $r(s_t,\cont_t) = r(s_t,\op_t(\disc_t,\cont_t) )$, $s_T \in S_G$ and
the generative model of the environment $T$ such that $s_{t+1} = T(s_t,\op_t(\disc_t,\cont_T))$

\section{Voronoi-based Optimistic Optimization for continuous space}
Suppose for now that the planning horizon is 1. This reduces to the black-box
function optimization problem. 

We would like to find the optimal action $x^*$ under the assumption that 
$$ f(x) - f(y) \leq \lambda \cdot d(x,y)$$
So, we have
$ f(x) \leq f(y) + \lambda \cdot d(x,y)$.

At any time point $t$, we know that we would have evaluated $x_1,\cdots x_t$ number of points.
We denote the Vornoi regions induced by these points as $V(x_i)$. By the definition of the 
Voronoi region, we know that the lowest upper-bound of value of points in a region
is provided by the furthest distance to the 
generator $x_i$ of the region. Hence, we  propose the following:
\begin{enumerate}
\item Compute the convex hulls of the Voronoi regions $V(x_1),\cdots, V(x_t)$. Denote $CH(V_i)$ as the
convex hull of the $i^{th}$ Voronoi region.
\item Select the most optimistic region by  $$ i^* = \arg\max_{i \in {1,\cdots,t}} f(x_i) + \lambda \cdot \max_{x \in CH(V_i)} d(x,x_i)$$ 
\item Randomly sample a point from $V(x_{i^*})$.
\end{enumerate}

\subsection{Planning with VOO}
\subsubsection{Monte-Carlo planning with progressive widening}

\begin{algorithm}[htb]
\small
   \caption{\sc{search}($s_0$)}
   \label{alg:search}
\begin{algorithmic}[1]
\STATE $\tree(s_0) = \{ N(s_0)=1,A=\emptyset,Q(s_0,\cdot)=0,N(s_0,\cdot)=0 \}$
\REPEAT
\STATE {\sc simulate}($s_0,0$)
\UNTIL{$timeout$}
\STATE{\bf return} $\argmax_{a\in\tree(s_0).A} \tree(s_0).Q(s,a)$
\end{algorithmic}
\end{algorithm}

\begin{algorithm}[htb]
\small
   \caption{\sc{simulate}($s, T, \delta, \alpha, \epsilon, S_G$)}
   \label{alg:simulate}
\begin{algorithmic}[1]
\IF{$s == infeasible$}
\STATE{\bf return} \text{infeasible-rwd}
\ELSIF{$s \in S_G$}
\STATE{\bf return} \text{success-rwd}
\ENDIF
\IF{$|\tree(s).A| \leq \delta \cdot \tree(s).N(s)$}
\STATE //progressive widening step
\STATE $a \sim ${\sc VOO}($\tree(s).A, \epsilon$)// \text{based on the algorithm above}
\STATE $\tree(s).A = \tree(s).A \cup a$
\ELSE
\STATE $a = \argmax_{a \in tree(s).A} Q(s,a, \alpha)$
\ENDIF
\STATE $s',r ~ \sim f(s,a)$
\STATE $R = r + \gamma \cdot $\sc{simulate}$(s',depth+1)$
\STATE $\tree(s).N(s,a) += 1$
\STATE $\tree(s).Q(s,a) = Q(s,a) + \frac{R-\tree(s).Q(s,a)}{\tree(s).N(s,a)}$
\end{algorithmic}
\end{algorithm}

Here, $\tree$ refers to the Monte Carlo tree, $N(s)$
is the number of times a state $s$ has been visited, $N(s,a)$ is
the number of times $a$ has been simulated from $s$, and $Q(s,a)$ is
the value of the action edge $a$ that goes out from $s$. $Q^+$ is the
augmented Q function that encourages exploration. $\pi$ is the stochastic
policy which we sample actions from, and $\delta$ is the progressive
widening parameter.

This Monte-Carlo planning approach is more advantageous
than heuristic-based search in that it is anytime, and the value
of the heuristic function improves overtime. It begins with a 
rather wrong estimate of the value function, 0, and the value
becomes more accurate as more simulations are performed.

There are subtleties that need to be taken care of when applying
this for TAMP problems. First, instead of depth, we should think
about detecting the dead-ends. However, detecting this is non-trivial.
Second, the function {\sc sampleaction} in line 10 of Algorithm 2 must take account
of the fact that there are infeasible actions. Unlike the 
game of Go where the infeasible moves are cheap and trivial to
detect, in TAMP, we need to call a motion planner to
find if the action is actually achievable. This is an expensive
step.

\subsubsection{A variant of A*}
The above approach is troublesome in that the function $Q$ is being
constantly updated, and that at a particular node, even if you choose
the same action you end up with different $Q$-value. Also, it has too
many parameters

We might naively try to optimize the entire sequence $\cont_1,\cdots,\cont_T$
directly, but doing so discards the important advantage of the search-based
method, which is that it has the ability to back-track, if deemed necessary.
We take the idea from A* to optimize $K^T = \cont_1,\cdots\cont_T$.

We will denote the $i^{th}$ sequence of actions whose length is $d \in [1,T]$ as
$K^{d}_i = \cont_1^{(i)} \cdots \cont_d^{(i)}$. Further, we assume that
the reward function is Lipschitz.

\begin{enumerate}
\item Inputs: {\sc VOO} parameter $\epsilon$, heuristic function $\hat{h}(K^*)$
\item Initialize the sequence set $S$ with an empty set.
\item for i=1 to timeout, 
\item Choose a sequence $K^{d} \in S$ according to
$$B(K^{d}) = \sum_{t=1}^{d-1} \gamma^t r(s_t,\cont_t) 
+ diameter(V(s_t.K))
 + \gamma^{d} \hat{h}(K^{d}) $$ 
\item Choose the dimension to cut, from $d+1$ choices, that
would reduce the upper bound $B(K^{d})$ the most.
\item Sample an action $a \sim ${\sc VOO}($\epsilon$), and divide the selected dimension
\item Add the new sequence to the sequence set S
\end{enumerate}

Alternatively, what happens if we ignore the diameter information? It will not
take account of how many actions we've s sampled at a particular dimension in
$K^T$.

The diameter information can also give insights into whether to switch
the plan skeleton. For example, if our $B(K^d)$ is bad to the point that
it does not contain the optimal value, then we should forgo it.


\section{Monte-Carlo planning algorithm for continuous domains}

\subsection{Dealing with infeasible actions}
if the next state node's state is equivalent to the current 
node's state, because the action failed,
then if I simulate forward from this state, then I might get a different reward.
If I get a high reward, then this might mean that current state is actually good, but
it does not mean that that particular (s,a), which got you back to the same state s, was good.
We would waste a lot of time updating the value of an edge that gets you back to the same state
What if I make it so that it returns 0 if it the sampled action is infeasible? That is, treat
(s,a) as a deadend-action. If I do this, then there is no recursion after this


\subsection{An idea about the roll-out policy}
It seems that the purpose of the roll-out policy is to supposed
to give a rough first estimate of the untravelled edge $(s,a)$.
We might do a (fast) roll-out by postponing the calls to the motion
planner when using this policy.

\section{Improving Monte-Carlo planning with learning}
Instead of initializing Q-values of new nodes with 0, we will
learn Q off-line and then soft-update it online. Also, instead
of using a uniform policy to sample actions, we will use
a learned policy. If our learned policy does not have a support
for the entire action space, then this will not guarantee probabilistic
completeness. So, our approach would be to use a uniform policy
with some small probability.

\section{Theoretical conjectures}
Suppose that our reward function is 1 at the goal state(s), and 0 otherwise.
In this case, finding a satisficing solution would be equivalent to finding an optimal solution.
A probabilistic completeness in this case would give us the guarantee
that we will find this optimal solution, if we use a uniform random
policy for $\pi$ as the number of evaluations of edges reaches infinity. 
This, however, does not give us the convergence rate, 
which is much more desirable.

Most of the papers in the Monte-carlo planning literature discusses about how to partition the
action space using a bandit algorithm, whether continuous or discrete,
 in order to show convergence rates. These algorithms, however, won't be able 
to scale to large dimensions. Hence our
hope is to use learning, and show the convergence rate as a function of
estimation error.
Here are specific questions :
\begin{enumerate}
\item Suppose when I sample an action (line 10 of Algorithm 2), I sample from a uniform
policy with probability $p$, and from a learned policy $\pi_\theta$. If my
learned policy has an error of $\delta$, $|\pi(s) - \pi^*(s)|\leq \delta\ \forall s \in S$,
then what is my convergence rate?
\item Suppose I have learned $Q_{\alpha}$ such that $|Q_{\alpha}(s,a)-Q^*(s,a)|\leq \delta\ \forall (s,a) \in S \times A$.
If in the backup step in line 18 of Algorithm 2, I use
$$ Q(s,a) \leftarrow (1-q)\cdot Q_{\alpha}(s,a) + q \cdot Q_{MC}(s,a)$$ 
where $Q_{MC}(s,a)$ is the value that I obtained by solely using
Monte-Carlo rollouts, and $q\in[0,1]$, then  what is my convergence rate?
\end{enumerate}

\section{Optimistic planning using the Voronoi region}
We would like to find the optimal action $x^*$ under the assumption that 
$$ f(x) - f(y) \leq \lambda \cdot d(x,y)$$
So, we have
$ f(x) \leq f(y) + \lambda \cdot d(x,y)$.

At any time point $t$, we know that we would have evaluated $x_1,\cdots x_t$ number of points.
We denote the Vornoi regions induced by these points as $V(x_i)$. By the definition of the 
Voronoi region, we know that the lowest upper-bound of value of points in a region
is provided by the furthest distance to the 
generator $x_i$ of the region. Hence, we  propose the following:
\begin{enumerate}
\item Compute the convex hulls of the Voronoi regions $V(x_1),\cdots, V(x_t)$. Denote $CH(V_i)$ as the
convex hull of the $i^{th}$ Voronoi region.
\item Select the most optimistic region by  $$ i^* = \arg\max_{i \in {1,\cdots,t}} f(x_i) + \max_{x \in CH(V_i)} d(x,x_i)$$ 
\item Randomly sample a point from $V(x_{i^*})$.
\end{enumerate}


\subsection{Computing the convex hull of the Voronoi region}
Suppose we have points $p_1,\cdots,p_n$. 
Denote the line segment from $p_i$ to $p_j$ as $\overline{p_ip_j}$. 
Given two points  $p_i$ and $p_j$, by the definition of a Voronoi region, we know that
the Voronoi regions whose sites are $p_i$ and $p_j$ are defined by 
the perpendicular bisector of the line segment $\overline{p_ip_j}$.
We can find this analytically. 


In general, if we have $n$ points, then the Voronoi region of a point $p_i$ can be 
determined by the intersection of the perpendicular bisectors of the line segment between 
 $p_i$ and all the other points. Finding the intersection of hyperplanes 
takes $O(n\log n)$ time, where $n$ is the number of hyperplanes. 
If we repeat this for all the points, then it takes $O(n^2\log n)$ time. 

Once we have the intersection, we can the set of points where the hyperplanes that define 
the intersection meet. These give the convex hull.

\subsection{Analysis}
Denote $\delta(V_i) = \max_{x \in CH(V_i)} d(x,x_i)$.
We know that if $i^{th}$ Voronoi region contains $\xopt$, then by our smoothness 
assumption it is bounded above by $$f(\xopt) \leq f(x_i) + \delta(V_i)$$
So, we will never evaluate a region whose upper bound is lower than $f(\xopt)$, because
$$ f(x_j) + \delta(V_j) \leq f(\xopt) \leq f(x_i) + \delta(V_i) $$

We can adopt the similar assumptions from DOO yet more precise. We know at at any point time
$t$, the diameters of the Voronoi region is decreasing. So, we have decreasing sequence
of diameters, with respect to the number of evaluations $t$, $B(t)$.


\section{Sampling-based approach}
This scales poorly with the number of dimensions. New algorithm:
\begin{enumerate}
\item With probability $\epsilon$, sample uniform randomly from $\mathcal{X}$
\item With probability $1-\epsilon$, sample uniform randomly from the best Voronoi region $V(x_{i^*})$, $ i^* = \arg\max_{i \in {1,\cdots,t}} f(x_i) $
\end{enumerate}


\section{Geometric task-and-motion-planning problem}
We denote the workspace of the robot as $\workspace$. A region $\region$ is
defined as a subset of the workspace. We define \emph{task constraint} as a set of objects 
 to be packed in the ordered sequence of regions,
$$[\region_1:(\obj^{(1)}_1,\cdots,\obj^{(1)}_{m_1}), \cdots, 
\region_T:(\obj^{(T)}_1,\cdots,\obj^{(T)}_{m_T})]$$ and the \emph{connecting region}
$\region_{connect}$ between two regions $\region_i$ and $\region_j$ as
$\region_{connect} \not\subset \region_i \cup \region_j$ such that,
 from any robot configuration in $\region_{connect}$, any configuration
in  $\region_i$ and $\region_j$ can be reached, 
without considering the movable obstacles. We will call the regions
specified in the task constraint as \emph{key regions}.


The problem statement is as follows. \emph{Given a task constraint, regions, and connecting 
regions between all pairs of regions, find the optimal motion plan to 
pack each object into its corresponding
key region in the given order, where the quality of a plan is measured by the number of
robot operations in it.}  We have following variability in our problem:
\begin{itemize}
\item The objects specified within a key region may or may not be ordered.
\item There may be other movable obstacles that are not specified in the task constraint.
\end{itemize}
We have following assumptions:
\begin{itemize}
\item We cannot find a feasible solution of the packing problem for $\region_{t}$ without
finding one for $\region_{t-1}$.
\item How we pack objects in $\region_{t-1}$ does not affect the
feasibility of the packing problem for $\region_{t}$, but affects the optimality.
\end{itemize}

Given this setup, G-TAMP problem is a sequence of interrelated packing problems. 
A packing problem
$$\mathcal{R}: (\obj_1,\cdots,\obj_{m})$$
consists of two parts: fetching objects from their initial regions, and
then placing them in $\region$. We assume that we cannot move a target
object once we pack it in $\region$, in order to remove redundant actions.

In fetching-and-placing a target object, we need to determine
objects that are in the way and clear them out. 
Therefore, we can see that a packing problem has
four unknowns, for each object:
\begin{enumerate}
\item Finding a robot reaching motion to the current target object
from the robot's current configuration. 
\item Finding a robot constraint-removal motion for clearing movable obstacles
from this reaching-motion
\item Finding a robot placing motion that places the current target
object in $\region$ such that all of target objects $\obj_1,\cdots,\obj_m$ can
be packed in $\region$.
\item Finding a robot constraint-removal motion for clearing movable obstacles
from this placing-motion
\end{enumerate}

These problems are related in that we cannot solve for 2 without solving for
1, 3 without solving for 2, and so on. However, if we tackle this naively in this 
order, we may waste our effort finding solutions 
for 1, 2, and 3 for many early objects, only to find 
that these early placements prevents a solution for latter objects. 


How should we define the sub-problem for which MCTS will be used for?
We should define it such that a solution to the earlier sub-problem
cannot have feasibility-level influence on the solution of the
next sub-problem, and we need the solution to the sub-problem
for the subsequent sub-problem. For example, we can separate object
fetching problem and NAMO problem, because how I fetch objects
cannot influence the feasibility of the NAMO problem, and
I cannot solve (or even define) the NAMO problem without knowing the fetching path.
But I cannot separate clearing each object within a NAMO problem
because if I clear an earlier obstacle such that it makes the
next obstacle unreachable, then the problem is infeasible.


I cannot define the reaching-problem for object $B_2$ if I have not
solved the reaching-and-constraint-removal motion for object $B_1$,
because I do not know how objects will be arranged. Also, I am assuming (?)
that no matter how I clear obstacles and reach-and-pack object $B_1$,
it cannot influence the feasiblity of reach-and-place of object $B_2$,
because at the end of clearing-reaching-and-packing, I will be at the
connecting region to $B_2$, because I ensure an existence of path
to a point in connecting region of $B_2$.

What's the connecting region between the box region and the home region?
What if I have to get stuff from the kitchen and shelf to box 1?




\bibliographystyle{IEEEtran}
\bibliography{references}
\end{document}
